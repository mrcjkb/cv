%\documentclass[12pt,a4paper,ngerman]{moderncv}
\documentclass[11pt,a4paper,ngerman,sans]{moderncv}
\usepackage[T1]{fontenc}
\usepackage[utf8]{inputenc}
\usepackage[T1]{fontenc}
% adjust the page margins
\usepackage[scale=0.85]{geometry}
\usepackage[scaled]{helvet} % Helvetica font
% \setmonofont{JetBrains Mono}[Contextuals=Alternate]
\usepackage{ragged2e}
\usepackage{relsize}
% customize the enumerate environments (i.e. enumerate, itemize, ...)
\usepackage{enumitem}
\usepackage{tikz}
\setlist{nolistsep}

% A custom version of the \cventry command that supports large itemized lists
% inside argument #7 (the custom cvitemize lists should be used!)
\newcommand*{\cventrylong}[7][.25em]{%
  \begin{tabular}{@{}p{\hintscolumnwidth}@{\hspace{\separatorcolumnwidth}}p{\maincolumnwidth}@{}}%
    \raggedleft\hintstyle{#2} &{%
        {\bfseries#3}%
        \ifthenelse{\equal{#4}{}}{}{, {\slshape#4}}%
        \ifthenelse{\equal{#5}{}}{}{, #5}%
        \ifthenelse{\equal{#6}{}}{}{, #6}%
    }%
  \end{tabular}%
  {\small#7}%
  \par\addvspace{#1}}

\tikzset{
    baseline,
    inner sep=2pt,
    minimum height=12pt,
    rounded corners=2pt  
}
\newcommand{\code}[1]{\mbox{% added this percent
    \ttfamily
    \tikz \node[anchor=base,fill=black!12]{#1};% added this percent
}}

% A custom version of the itemize environment that sets the appropriate left
% margin for use inside \cventylong
\newlist{cvitemize}{itemize}{1}
\setlist[cvitemize]{label=\labelitemi,%
leftmargin=\hintscolumnwidth+\separatorcolumnwidth+\labelwidth+\labelsep}

\setcounter{secnumdepth}{0}
\setlength{\parskip}{\medskipamount}
\setlength{\parindent}{0pt}

\makeatletter

%%%%%%%%%%%%%%%%%%%%%%%%%%%%%% LyX specific LaTeX commands.
\special{papersize=\the\paperwidth,\the\paperheight}
\newcommand{\noun}[1]{\textsc{#1}}
%%%%%%%%%%%%%%%%%%%%%%%%%%%%%% User specified LaTeX commands.

%C++
% \newcommand\CC{C\nolinebreak\hspace{-.05em}\raisebox{.4ex}{\relsize{-3}{\textbf{+}}}\nolinebreak\hspace{-.10em}\raisebox{.4ex}{\relsize{-3}{\textbf{+}}}}
\newcommand\CC{\texttt{C}\nolinebreak[4]\hspace{-.05em}\raisebox{.3ex}{\relsize{-3}{\textbf{++}}}}
\newcommand\C{\texttt{C}\nolinebreak[4]\hspace{-.05em}\raisebox{.3ex}{\relsize{-3}}}

\renewcommand\familydefault{\sfdefault} 

%\definecolor{porange}{rgb}{0.9333, 0.6352, 0.2118}
% required
\moderncvcolor{mrc}
\moderncvtheme[mrc]{classic}
%\moderncvtheme[grey]{classic}
% possible themes are "classic" and "casual"
% optional argument are 'blue' (default), 'orange', 'red', 'green', 'grey' and 'roman' (for roman fonts, instead of sans serif fonts)

% required
\firstname{Marc}
% required
\familyname{Jakobi}

% optional, remove the line if not wanted
% \title{born 1989-04-18 in Filderstadt}
\title{Curriculum Vitae}

% optional
% \address{street and number}{postcode city}
% '\\' adds a line break
\address{Winterthur, CH}

% optional
%\phone{+49 (0)30 XXXX XXXX} 
% \mobile{<redacted>}
\email{marc@jakobi.dev}
\extrainfo{\href{https://github.com/mrcjkb}{github.com/mrcjkb}\\ \href{https://mrcjkb.dev}{mrcjkb.dev}}

% optional
% \photo[height]{name}
% 'height' is the height the picture is resized to
% 'name' is the name of the picture file
\photo[70pt]{CV_photo.jpg}

% optional
\quote{<<Insanity is running the same build over and over again and getting different results>>}

\makeatother
\begin{document}

% ------------- Cover letter --------------

% \clearpage

% \recipient{HR Department}{Sympower\\Prinsengracht 437-A\\1016 HM – Amsterdam, The Netherlands} % Letter recipient
% \date{\today} % Letter date
% \opening{Dear Sir or Madam,} % Opening greeting
% \closing{Sincerely yours,} % Closing phrase
% \enclosure[Attached]{curriculum vit\ae{}} % List of enclosed documents

% \makelettertitle % Print letter title
% % \justify
% NOTE: cover letter placeholder
% I was recently contacted regarding a job opening at ...
% \section{About me}
% \section{Why xyz?}
% \makeletterclosing % Print letter signature

% \newpage

% ------------------- CV -------------------

\maketitle

\cvitem{}{
Experienced software engineer and open source advocate with a strong background
in distributed systems, renewable energy, and FOSS community leadership.
Proven track record in backend architecture, CI/CD automation, and impactful open source contributions.
Passionate about functional programming, test-driven development, reproducible builds,
and bringing innovation to sustainable technologies.
}

\section{Professional career}

\cventry{2022\textendash current}{Backend developer}{tiko Energy Solutions AG}{}{} 
{
  Architected and developed distributed systems primarily in \href{https://www.haskell.org/}{Haskell},
  leveraging \href{https://nixos.org/}{Nix} for reproducible builds.
  (80 \% workload since 2023)\newline
  \\
  \textbf{\emph{Virtual Power Plant and IoT devices}}
  \begin{itemize}
    \item Microservices written in Haskell for high-throughput, low-latency device management
          and ancillary services (FCR + aFRR)
    \item Legacy monolith written in Java 21 + Scala 2.13
    \item Redis/Valkey, Kafka, CBOR, MQTT, PostgreSQL, Hazelcast, GraphQL
    \\
  \end{itemize}
	\textbf{\emph{Data pipelines}}
  \begin{itemize}
    \item Resilient event-driven pipelines written in Haskell
    \item MQTT, Kafka, Thrift, Protobuf, Avro, rocksDB-cloud, TimescaleDB, RabbitMQ
    \\
  \end{itemize}
	\textbf{\emph{Web services}}
  \begin{itemize}
    \item Built and maintained scalable Haskell APIs (servant, wai/warp)
    \\
  \end{itemize}
	\textbf{\emph{Quality \& test automation}}
  \begin{itemize}
    \item Propagated behaviour driven development practices,
          increasing integration test coverage.
    \item Designed KVM-based integration tests of distributed systems
          using the \code{nixosTest}\newline framework,
          enabling zero rollbacks in VPP service deployments.
    \item Led efforts to improve the build system, organise and unify the codebase,
          write and update documentation, and keep dependencies up to date.
    \\
  \end{itemize}
	\textbf{\emph{Continuous integration, deployment}}
  \begin{itemize}
    \item Hydra, Morph, Colmena, GitLab, k8s, ArgoCD, Kustomize,\newline Terraform, AWS,
          SOPS, Dhall
    \item Release management, regular system deployments
    \\
  \end{itemize}
	\textbf{\emph{API architecture and collaboration}}
  \begin{itemize}
    \item Worked with Data Science, Full Stack and Firmware teams to design cross-system APIs.
    \item Mentored, supported and pair-programmed with junior backend team members.
    \\
  \end{itemize}
	\textbf{\emph{Monitoring, Alerting}}
  \begin{itemize}
    \item Implemented OpenTelemetry instrumentation, enabling observability across services.
    \item Prometheus, Grafana, Zabbix, PagerDuty, Elasticsearch, Kibana
    \item On-call rotation, incident response, root cause analysis
    \\
  \end{itemize}
}

\cventry{2023\textendash current}{Open source volunteer}{20 \%}{}{} 
{
  Reduced workload to 80 \% in 2023 to dedicate time to maintaining and contributing to FOSS. Selected projects:
  \newline
  \newline
  \textbf{\emph{\href{https://opencollective.com/lumen-labs}{Lumen Labs}}}
  \begin{itemize}
    \item Lead maintainer of the \href{https://lux.lumen-labs.org}{\code{Lux} package manager} for Lua (written in Rust)
    \\
  \end{itemize}
  \textbf{\emph{\href{https://nixos.org/}{NixOS}}}
  \begin{itemize}
    \item Maintainer of various packages and NixOS modules
    \item Member of the \href{https://github.com/orgs/NixOS/teams/neovim}{Neovim} and
          \href{https://github.com/orgs/NixOS/teams/lua}{Lua} maintainer teams
    \\
  \end{itemize}
  \textbf{\emph{\href{https://neovim.io/}{Neovim}}}
  \begin{itemize}
    \item Core contributions
    \item Maintainer of various popular plugins and GitHub actions workflows, for example:
      \begin{itemize}
        \item \href{https://github.com/mrcjkb/rustaceanvim}{rustaceanvim (Rust)}
        \item \href{https://github.com/mrcjkb/haskell-tools.nvim}{haskell-tools.nvim}
        \item \href{https://github.com/mrcjkb/neotest-haskell}{neotest-haskell}
        \item \href{https://github.com/nvim-neorocks/rocks.nvim}{rocks.nvim}
        \item \href{https://github.com/mrcjkb/kickstart-nix.nvim}{kickstart-nix.nvim}
        \item \href{https://github.com/nvim-neorocks/luarocks-tag-release}{luarocks-tag-release}\newline
      \end{itemize}
  \end{itemize}
}
{}

\cventry{2017\textendash 2022}{Software engineer}{Vela Solaris AG}{}{}
{
  Java 17, Kotlin, Docker Swarm, Gradle\newline
  \\
	\textbf{\emph{Polysun Simulation Software}} \emph{(desktop app written in Java)}
  \begin{itemize}
    \item Simulation models: Batteries, PV/PVT, controllers, thermal components\newline
      (heat pumps, storage tanks, co-generators, etc.), eMobility.\newline
      \emph{
       Examples: Implementation of a new battery model;\newline
       control algorithms for PV, batteries and heat pumps.\newline
       Improvement of existing models.
      }
    \item Plugins (e.g. for co-simulations with Python/Matlab/Simulink/PLC).
    \item Charting, reporting.
    \item UI.
    \item Code reviews (Bitbucket).
    \item Publications/Presentations/Advanced workshops (e.g. at conferences).
      \\
  \end{itemize}
	\textbf{\emph{Polysun BIM}} 
	\emph{Description: Automation of engineering workflows in the building sector;\newline
	data exchange with other applications.}\newline
  \begin{itemize}
    \item Customer interviews / requirement analysis.
    \item Rapid prototyping (interactive mock-ups) + concept validation with pilot customers.
    \item Roadmap planning.
    \item Software engineering/development:
      \begin{itemize}
        \item \emph{Desktop app with a modular bundle architecture}\newline
          (Java, Kotlin, Hibernate (H2), Vavr, ReactiveX, Protobuf, ArchUnit).
        \item \emph{Cloud platform (backend), microservice architecture}\newline
          (Spring Boot 2, Kafka Streams, MongoDB, Graphwalker).
          \\
      \end{itemize}
  \end{itemize}
  \textbf{\emph{Infrastructure}}
  \begin{itemize}
    \item CI/CD: Jenkins, Docker Swarm on AWS, e2e testing, release automation
    \item Internal cloud infrastructure (e.g., license migration, sales automation, ...)\newline
      \emph{Apache Camel}.	
    \item Development of internal CLI applications (Haskell, Python)
      \\
  \end{itemize}
}
{}


\cventry{
  2014\textendash 2017}{Research assistant}
  {Projects:
    \href{https://pvspeicher.htw-berlin.de/forschungsprojekte/pv-store/}{PVstore},
    \href{https://pvspeicher.htw-berlin.de/forschungsprojekte/twinpower/}{TwinPower}
  }{HTW Berlin}{}
  {Optimisation of PV systems with batteries and heat pumps.\newline
  Implementation of simulation models in Matlab.
  }{}

\cventry{2013\textendash 2014}{Research assistant \& intern}{\href{https://pvspeicher.htw-berlin.de/forschungsprojekte/pvprog/}
{Project: PVprog}}
{HTW Berlin}{}
{Development of forecast-based operational strategies for PV storage systems.}
{}

\clearpage
\section{Education}

\cventry{2016\textendash 2017}{Master of Science - Renewable Energy Systems}{Hochschule für Technik und Wirtschaft}{HTW Berlin}{}{GPA: A (with honours), HTW Berlin.}{}

\cventry{2012\textendash 2016}{Bachelor of Science - Renewable Energy Systems}{Hochschule für Technik und Wirtschaft}{HTW Berlin}{}{GPA: A (with honours), HTW Berlin.}{}

\section{\href{https://pvspeicher.htw-berlin.de/wp-content/uploads/2017/11/Jakobi-2017-Development-of-model-based-control-applications-compliant-with-IEC-61499.pdf}{Master's Thesis}}

\cvitem{Title}{\emph{\href{https://pvspeicher.htw-berlin.de/wp-content/uploads/2017/11/Jakobi-2017-Development-of-model-based-control-applications-compliant-with-IEC-61499.pdf}{Development of model-based control applications compliant with IEC~61499 for building energy systems with a focus on photovoltaics}}}

\vspace{0cm}


\cvitem{Supervision}{Prof. Dr.-Ing. Volker Quaschning, M. Sc. Tjarko Tjaden}

\cvitem{Grade}{Thesis and oral examination: 1.0 / studies: 1.2\newline(GPA: A)}

\cvitem{Short summary}{Development of intelligent control applications for PV, battery and heat pump systems in compliance with IEC~61499. Development of communication interfaces for simulation tools such as Polysun and Matlab. Validation via co-simulations. Extension of the runtime environment (4diac-RTE) with a REST communication interface and set-up of a field test.}
\section{\href{https://pvspeicher.htw-berlin.de/wp-content/uploads/2014/05/JAKOBI-2015-Optimierung-des-Netzeinspeiseverhaltens-von-deutschlandweit-verteilten-PV-Speichersytemen-mit-prognosebasierten-Betriebsstrategien.pdf}{Bachelor's Thesis}}

\cvitem{Title}{\emph{\href{https://pvspeicher.htw-berlin.de/wp-content/uploads/2014/05/JAKOBI-2015-Optimierung-des-Netzeinspeiseverhaltens-von-deutschlandweit-verteilten-PV-Speichersytemen-mit-prognosebasierten-Betriebsstrategien.pdf}{Optimierung der Netzeinspeisung von deutschlandweit verteilten PV-Speichersystemen mit prognosebasierten Betriebsstrategien}}}

\vspace{0cm}

\cvitem{Supervision}{Prof. Dr.-Ing. Volker Quaschning, M. Sc. Johannes Weniger}

\cvitem{Grade}{Thesis and oral examination: 1.0 / studies: 1.3\newline(GPA: A)}

\cvitem{Short summary}{Modelling and CUDA-simulation of 46126 Germany-wide distributed PV storage systems. Examination of the cumulated influence of various operational strategies and optimisation of forecasting- and control algorithms regarding self-sufficiency and grid integration.}

\section{Languages}

\cvitem{English}{Mother tongue}

\cvitem{German}{Mother tongue}

\cvitem{Swiss German}{Good comprehension (not spoken)}

\cvitem{French}{B1}

\clearpage
\section{Professional skills}

\cvitem{Programming languages}{
  \texttt{Haskell},
  \texttt{Nix},
  \texttt{Rust},
  \texttt{Elixir},
  \C,
  \texttt{Scala},
  \texttt{Kotlin},
  \texttt{Java},
  \texttt{Lua},
  \texttt{Python},
  \CC,
  \texttt{TypeScript},
  \texttt{MATLAB}
}

\cvitem{Software engineering}{
  Functional software architecture,
  test-driven development,
  behaviour-driven development,
  pair programming,
  distributed systems,
  CI/CD pipelines,
  release automation
}

\cvitemwithcomment{Development environment}
  {Neovim, NixOS, tmux, Nushell, jj-vcs}{UNIX is my IDE}

\cvitem{Simulation}{Polysun, Simulink, TRNSYS}

\cvitem{Scientific documentation}{\noun{\LaTeX{}}, pandoc}

\section{Professional interests}

\cvitem{Renewable energy}{Photovoltaics, batteries, thermal systems, eMobility, sector coupling, energy management}

\cvitem{Software development}{
  FOSS/Linux development,
  Trade literature,
  Hackathons,
  SoCraTes unconferences,
  Ensemble programming,
}

\section{Leisure-time activities}

\cvitem{Sports}{{\small{}Running, cycling, hiking}}
\cvitem{Cooking}{}
\cvitem{Gardening}{}


\section{Publications}

\cvitem{Conference paper}{Jakobi, M.; Kunath, L.; Witzig, A. \emph{\href{https://proceedings.ises.org/paper/eurosun2018/eurosun2018-0147-Jakobi.pdf}{BIM use-case: Model-based performance optimization.}} EuroSun international conference on solar energy for buildings and industry, Rapperswil, 2018.}

\cvitem{Conference paper}{Jakobi, M.; Stöckli, U.; Tjaden, T.; Quaschning, Q. \emph{\href{http://proceedings.ises.org/paper/eurosun2018/eurosun2018-0151-Jakobi.pdf}{From simulation to reality: IEC 61499 compliant control applications for solar energy systems.}} EuroSun international conference on solar energy for buildings and industry, Rapperswil, 2018.}

\cvitem{Conference paper}{Jakobi, M.; Stöckli, U.; Tjaden, T.; Quaschning, Q. \emph{\href{https://www.velasolaris.com/wp-content/uploads/2018/12/20180504_publikation_symposiumpv18_jakobi.pdf}{Von der Simulation zur Realität: IEC 61499 konforme Regelanwendungen für Solare Energiesysteme.}} Symposium photovoltaische Solarenergie, Bad Staffelstein, 2018.}

\cvitem{Thesis}{Jakobi, M.: \emph{\href{https://pvspeicher.htw-berlin.de/wp-content/uploads/2017/11/Jakobi-2017-Development-of-model-based-control-applications-compliant-with-IEC-61499.pdf}{Development of model-based control applications compliant with IEC~61499 for building energy systems with a focus on photovoltaics.}} Master's thesis, Hochschule für
	Technik und Wirtschaft, Berlin, 2017.}

\cvitem{Thesis}{Jakobi, M.: \emph{\href{http://pvspeicher.htw-berlin.de/wp-content/uploads/2014/05/JAKOBI-2015-Optimierung-des-Netzeinspeiseverhaltens-von-deutschlandweit-verteilten-PV-Speichersytemen-mit-prognosebasierten-Betriebsstrategien.pdf}{\textit{Optimierung des Netzeinspeiseverhaltens von deutschlandweit verteilten PV-Speichersystemen mit prognosebasierten Betriebsstrategien.}}} Bachelor's thesis, Hochschule für
Technik und Wirtschaft, Berlin, 2016.}

\cvitem{Data \& code}{Jakobi, M.; Schmidt, M.; Anyangbe, F. \emph{\href{https://github.com/MrcJkb/lfpBattery/blob/master/Documentation/lfpBattery_Documentation.pdf}{Cell resolved Matlab OOP model of a lithium iron phosphate battery pack}}, TU Berlin, 2017.}

\cvitem{Co-author}{Weniger, J.; Bergner, J.; Beier, D.; Jakobi, M.; Tjaden, T.; Quaschning,
Q.: \emph{\href{https://pvspeicher.htw-berlin.de/wp-content/uploads/2014/04/WENIGER-2015-Grid-Feed-in-Behavior-of-Distributed-PV-Battery-Systems.pdf}{Grid Feed-in Behavior of Distributed PV Battery Systems}}.
30th European Photovoltaic Solar Energy Conference and Exhibition,
Hamburg, 2015. }

\end{document}
