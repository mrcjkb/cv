%\documentclass[12pt,a4paper,ngerman]{moderncv}
\documentclass[11pt,a4paper,ngerman,sans]{moderncv}
\usepackage[T1]{fontenc}
\usepackage[utf8]{inputenc}
\usepackage[T1]{fontenc}
% adjust the page margins
\usepackage[scale=0.85]{geometry}
\usepackage[scaled]{helvet} % Helvetica font
% \setmonofont{JetBrains Mono}[Contextuals=Alternate]
\usepackage{ragged2e}
\usepackage{relsize}
% customize the enumerate environments (i.e. enumerate, itemize, ...)
\usepackage{enumitem}
\usepackage{tikz}
\setlist{nolistsep}

% A custom version of the \cventry command that supports large itemized lists
% inside argument #7 (the custom cvitemize lists should be used!)
\newcommand*{\cventrylong}[7][.25em]{%
  \begin{tabular}{@{}p{\hintscolumnwidth}@{\hspace{\separatorcolumnwidth}}p{\maincolumnwidth}@{}}%
    \raggedleft\hintstyle{#2} &{%
        {\bfseries#3}%
        \ifthenelse{\equal{#4}{}}{}{, {\slshape#4}}%
        \ifthenelse{\equal{#5}{}}{}{, #5}%
        \ifthenelse{\equal{#6}{}}{}{, #6}%
    }%
  \end{tabular}%
  {\small#7}%
  \par\addvspace{#1}}

\tikzset{
    baseline,
    inner sep=2pt,
    minimum height=12pt,
    rounded corners=2pt  
}
\newcommand{\code}[1]{\mbox{% added this percent
    \ttfamily
    \tikz \node[anchor=base,fill=black!12]{#1};% added this percent
}}

% A custom version of the itemize environment that sets the appropriate left
% margin for use inside \cventylong
\newlist{cvitemize}{itemize}{1}
\setlist[cvitemize]{label=\labelitemi,%
leftmargin=\hintscolumnwidth+\separatorcolumnwidth+\labelwidth+\labelsep}

\setcounter{secnumdepth}{0}
\setlength{\parskip}{\medskipamount}
\setlength{\parindent}{0pt}

\makeatletter

%%%%%%%%%%%%%%%%%%%%%%%%%%%%%% LyX specific LaTeX commands.
\special{papersize=\the\paperwidth,\the\paperheight}
\newcommand{\noun}[1]{\textsc{#1}}
%%%%%%%%%%%%%%%%%%%%%%%%%%%%%% User specified LaTeX commands.

%C++
% \newcommand\CC{C\nolinebreak\hspace{-.05em}\raisebox{.4ex}{\relsize{-3}{\textbf{+}}}\nolinebreak\hspace{-.10em}\raisebox{.4ex}{\relsize{-3}{\textbf{+}}}}
\newcommand\CC{\texttt{C}\nolinebreak[4]\hspace{-.05em}\raisebox{.3ex}{\relsize{-3}{\textbf{++}}}}
\newcommand\C{\texttt{C}\nolinebreak[4]\hspace{-.05em}\raisebox{.3ex}{\relsize{-3}}}

\renewcommand\familydefault{\sfdefault} 

%\definecolor{porange}{rgb}{0.9333, 0.6352, 0.2118}
% required
\moderncvcolor{mrc}
\moderncvtheme[mrc]{classic}
%\moderncvtheme[grey]{classic}
% possible themes are "classic" and "casual"
% optional argument are 'blue' (default), 'orange', 'red', 'green', 'grey' and 'roman' (for roman fonts, instead of sans serif fonts)

% required
\firstname{Marc}
% required
\familyname{Jakobi}

% optional, remove the line if not wanted
% \title{born 1989-04-18 in Filderstadt}
\title{Curriculum Vitae}

% optional
% \address{street and number}{postcode city}
% '\\' adds a line break
\address{Etzbergstrasse 15}{8405 Winterthur, Switzerland}

% optional
%\phone{+49 (0)30 XXXX XXXX} 
\mobile{+41 (0)79 458 04 77}
\email{marc@jakobi.dev}
\extrainfo{github.com/mrcjkb\\ mrcjkb.dev}

% optional
% \photo[height]{name}
% 'height' is the height the picture is resized to
% 'name' is the name of the picture file
\photo[70pt]{CV_photo.jpg}

% optional
\quote{Insanity is running the same Nix build over and over again and expecting different results.}

\makeatother
\begin{document}

% ------------- Cover letter --------------

% \clearpage

% \recipient{HR Department}{Sympower\\Prinsengracht 437-A\\1016 HM – Amsterdam, The Netherlands} % Letter recipient
% \date{\today} % Letter date
% \opening{Dear Sir or Madam,} % Opening greeting
% \closing{Sincerely yours,} % Closing phrase
% \enclosure[Attached]{curriculum vit\ae{}} % List of enclosed documents

% \makelettertitle % Print letter title
% % \justify
% XXX: Very outdated
% I was recently contacted regarding a job opening at ...
% \section{About me}
% My educational background is in renewable energies. During my Master's degree, I specialised in solar PV, battery storage, and sector coupling (renewable thermal energy and eMobility).
% Early on in my studies, I started working part-time in a research group, with a focus on forecasting, energy management, and simulation (primarily using \texttt{Matlab}). 
% It was then that I discovered my passion for software development, and started diving deeper, using literature to teach myself various programming languages and design patterns. 
% After my studies, I started my current job: Developing simulation software for sector-coupled renewable energy systems (desktop and SaaS applications).
% As we are a small team, my responsibilities cover the full stack: 
% Requirement analysis, design, rapid prototyping, front and back end development, code reviews, CI/CD, 3rd level support, to name a few. 
% With the skill set I have gathered, my team members now consider me a senior developer. 
% \newline
% At work, I mostly use \texttt{Java} and \texttt{Python}, with a little bit of \texttt{Kotlin}. 
% Privately, I prefer \texttt{Haskell} and \texttt{Rust}. My coding style is mostly functional (fluent and monadic) and partly test-driven, with a high degree of automation in enforcing architectural rules (where practical). 
% Although I currently develop proprietary software professionally, I am a strong advocate for FOSS, and sharing libraries with the community whenever possible. 
% Despite being passionate about my work, and educating myself with literature daily,  work life balance is very important to me. 
% To ensure my physical and mental well-being, I adhere to a strict sports schedule.  
% \section{Why Sympower?}
% I already consider myself very lucky to be able to combine two of my passions professionally. However, I do feel the desire for greater challenges, and a greater degree of freedom in choosing my tech stack. Therefore, I am open to new opportunities. What has motivated me to send this application to you is my strong passion for renewable energies and the fact that \emph{I want to continue working in this field.}. Since high level software engineering is rather rare in that domain, I do not want to miss out on the opportunity to at least get to know you.
% \newline
% That being said, due to the fact that my wife and I live and work in Switzerland, my salary expectations are higher than what is common in the EU: In the range of EUR~90..100k, based on 33~h/week.
% If you too think that we could be a great match, and would like to get to know me, please do not hesitate to contact me.
% \newline
% \newline
% \makeletterclosing % Print letter signature

% \newpage

% ------------------- CV -------------------

\maketitle

\section{Professional career}

\cventry{2022\textendash current}{Backend developer}{tiko Energy Solutions AG}{}{} 
{
  Microservices mostly written in Haskell, built with Nix.\newline
  (80 \% since 2023)\newline
  \\
  \textbf{\emph{Virtual power plant and IOT devices}}
  \begin{itemize}
    \item Redis, Kafka, CBOR, MQTT, PostgreSQL, Hazelcast, GraphQL
    \item Microservices written in Haskell
    \item Legacy monolith written in Java 21 + Scala 2.13
    \\
  \end{itemize}
	\textbf{\emph{Data pipelines}}
  \begin{itemize}
    \item MQTT, Kafka, Thrift, Protobuf, Avro, rocksDB-cloud, TimescaleDB
    \item Microservices written in Haskell
    \\
  \end{itemize}
	\textbf{\emph{Web services}}
  \begin{itemize}
    \item Haskell (servant, wai/warp)
    \\
  \end{itemize}
	\textbf{\emph{CI, Deployment}}
  \begin{itemize}
    \item Hydra, Morph, GitLab, Kubernetes, ArgoCD, Kustomize,\newline Terraform, AWS, SOPS, Dhall
    \item Release management
    \\
  \end{itemize}
	\textbf{\emph{Quality}}
  \begin{itemize}
    \item Propagated behaviour driven development practices.
    \item Introduced and developed KVM-based integration tests using the \code{nixosTest} framework.
    \\
  \end{itemize}
	\textbf{\emph{Monitoring, Alerting}}
  \begin{itemize}
    \item Prometheus, Zabbix, Elasticsearch, OpenTelemetry, Kibana
    \item Oncall, Incident response
    \\
  \end{itemize}
}

\cventry{2023\textendash current}{Open source volunteer}{20 \%}{}{} 
{
  I spend 20 \% of my work week maintaining and contributing to various open source projects, e.g.:\newline
  \newline
  \textbf{\emph{NixOS}}
  \begin{itemize}
    \item Maintainer of various packages and NixOS modules
    \item Co-maintainer of the Lua and Vim/Neovim ecosystems
    \\
  \end{itemize}
  \textbf{\emph{Nvim-neorocks}}
  \begin{itemize}
    \item Co-maintainer of the \code{Lux} package manager for Lua
    \\
  \end{itemize}
  \textbf{\emph{Neovim}}
  \begin{itemize}
    \item Core contributions
    \item Maintainer of various plugins for Haskell, Nix and Rust development
  \end{itemize}
}
{}

\cventry{2017\textendash 2022}{Software engineer}{Vela Solaris AG}{}{}
{
  Java 17, Kotlin, Docker Swarm, Gradle\newline
  \\
	\textbf{\emph{Polysun Simulation Software}} \emph{(desktop app written in \texttt{Java})}
  \begin{itemize}
    \item Simulation models: Batteries, PV/PVT, controllers, thermal components\newline
      (heat pumps, storage tanks, co-generators, etc.), eMobility.\newline
      \emph{
       Examples: Implementation of a new battery model;\newline
       control algorithms for PV, batteries and heat pumps.\newline
       Improvement of existing models.
      }
    \item Plugins (e.g. for co-simulations with \texttt{Python}/\texttt{Matlab/Simulink}/PLC).
    \item Charting, reporting.
    \item UI. 
    \item Code reviews (Bitbucket).
    \item Publications/Presentations/Advanced workshops (e.g. at conferences).
      \\
  \end{itemize}
	\textbf{\emph{Polysun BIM}} 
	\emph{Description: Automation of engineering workflows in the building sector;\newline
	data exchange with other applications.}\newline
  \begin{itemize}
    \item Customer interviews / requirement analysis.
    \item Rapid prototyping (interactive mock-ups) + concept validation with pilot customers.
    \item Roadmap planning.
    \item Software engineering/development:
      \begin{itemize}
        \item \emph{Desktop app with a modular bundle architecture}\newline
          (\texttt{Java}, \texttt{Kotlin}, Hibernate (H2), Vavr, ReactiveX, Protobuf, ArchUnit).
        \item \emph{Cloud platform (backend), microservice architecture}\newline
          (Spring Boot 2, Kafka Streams, MongoDB, Graphwalker).
          \\
      \end{itemize}
  \end{itemize}
  \textbf{\emph{Infrastructure}}
  \begin{itemize}
    \item CI/CD: Jenkins, Docker Swarm on AWS, e2e testing, release automation
    \item Internal cloud infrastructure (e.g., license migration, sales automation, ...)\newline
      \emph{Apache Camel}.	
    \item Development of internal CLI applications (\texttt{Haskell}, \texttt{Python})
      \\
  \end{itemize}
}
{}


\cventry{
  2014\textendash 2017}{Research assistant}
  {Projects: 
    \href{https://pvspeicher.htw-berlin.de/forschungsprojekte/pv-store/}{PVstore},
    \href{https://pvspeicher.htw-berlin.de/forschungsprojekte/twinpower/}{TwinPower}
  }{HTW Berlin}{}
  {Optimisation of PV systems with batteries and heat pumps.\newline
  Implementation of simulation models in \texttt{Matlab}.
  }{}

\cventry{2013\textendash 2014}{Research assistant \& intern}{\href{https://pvspeicher.htw-berlin.de/forschungsprojekte/pvprog/}
{Project: PVprog}}
{HTW Berlin}{}
{Development of forecast-based operational strategies for PV storage systems.}
{}

\clearpage
\section{Education}

\cventry{2016\textendash 2017}{Master of Science - Renewable Energy Systems}{Hochschule für Technik und Wirtschaft}{HTW Berlin}{}{Completion of Master's Thesis in the research group ''solar storage systems'' (Thesis and oral examination: 1.0 / studies: 1.2 [with honours]), HTW Berlin.}{}

\cventry{2012\textendash 2016}{Bachelor of Science - Renewable Energy Systems}{Hochschule für Technik und Wirtschaft}{HTW Berlin}{}{Completion of Bachelor's Thesis in the research group ''solar storage systems'' (Thesis and oral examination: 1.0 / studies: 1.3 [with honours]), HTW Berlin.}{}

\cventry{2006\textendash 2009}{A levels}{Freie Waldorfschule Saar-Hunsrück}{Walhausen}{Grade: 2.1}{}{}

\section{\href{https://pvspeicher.htw-berlin.de/wp-content/uploads/2017/11/Jakobi-2017-Development-of-model-based-control-applications-compliant-with-IEC-61499.pdf}{Master's Thesis}}

\cvitem{Title}{\emph{\href{https://pvspeicher.htw-berlin.de/wp-content/uploads/2017/11/Jakobi-2017-Development-of-model-based-control-applications-compliant-with-IEC-61499.pdf}{Development of model-based control applications compliant with IEC~61499 for building energy systems with a focus on photovoltaics}}}

\vspace{0cm}


\cvitem{Supervision}{Prof. Dr.-Ing. Volker Quaschning, M. Sc. Tjarko Tjaden}

\cvitem{Grade}{Thesis and oral examination: 1.0 / studies: 1.2}

\cvitem{Short summary}{Development of intelligent control applications for PV, battery and heat pump systems in compliance with IEC~61499. Development of communication interfaces for simulation tools such as Polysun and Matlab. Validation via co-simulations. Extension of the runtime environment (4diac-RTE) with a REST communication interface and set-up of a field test.}
\section{\href{https://pvspeicher.htw-berlin.de/wp-content/uploads/2014/05/JAKOBI-2015-Optimierung-des-Netzeinspeiseverhaltens-von-deutschlandweit-verteilten-PV-Speichersytemen-mit-prognosebasierten-Betriebsstrategien.pdf}{Bachelor's Thesis}}

\cvitem{Title}{\emph{\href{https://pvspeicher.htw-berlin.de/wp-content/uploads/2014/05/JAKOBI-2015-Optimierung-des-Netzeinspeiseverhaltens-von-deutschlandweit-verteilten-PV-Speichersytemen-mit-prognosebasierten-Betriebsstrategien.pdf}{Optimierung der Netzeinspeisung von deutschlandweit verteilten PV-Speichersystemen mit prognosebasierten Betriebsstrategien}}}

\vspace{0cm}

\cvitem{Supervision}{Prof. Dr.-Ing. Volker Quaschning, M. Sc. Johannes Weniger}

\cvitem{Grade}{Thesis and oral examination: 1.0 / studies: 1.3}

\cvitem{Short summary}{Modelling and CUDA-simulation of 46126 Germany-wide distributed PV storage systems. Examination of the cumulated influence of various operational strategies and optimisation of forecasting- and control algorithms regarding self-sufficiency and grid integration.}

\section{Other}

\cventry{2016}{Intern/project work}{}{Vela Solaris AG}{Winterthur}{Implementation and validation of the "PVprog" forcast-based control algorithm in Polysun.}

\cventry{2011}{Temp}{}{Brose Fahrzeugteile GmbH; Leuwico GmbH}{Coburg}{Assembly, manufacture}

\cventry{2011}{Intern}{}{Lasco Umformtechnik GmbH}{Coburg}{Montage, inspection, mechanical workshop}

\cventry{2009\textendash 2010}{Social service}{}{REHA GmbH}{Neunkirchen}{}

\section{Languages}

\cvitemwithcomment{English}{Mother tongue}{}

\cvitemwithcomment{German}{Mother tongue}{}

\cvitemwithcomment{French}{Good knowledge}{School, grades 7-13; regular. visits to France and Morocco}

\clearpage
\section{Professional skills}

\cvitemwithcomment{Programming languages}{
  \texttt{Haskell},
  \texttt{Nix},
  \texttt{Rust},
  \C,
  \texttt{Scala},
  \texttt{Kotlin},
  \texttt{Java},
  \texttt{Lua},
  \texttt{Python},
  \CC
}{}

\cvitemwithcomment{Development environment}{
  \texttt{Neovim},
  \texttt{NixOS},
  \texttt{tmux},
  \texttt{Nushell},
  \texttt{coreutils}
}{I prefer to work in the terminal}

\cvitemwithcomment{Simulation}{Polysun, Simulink, TRNSYS}{}{}

\cvitem{Scientific documentation}{\noun{\LaTeX{}}, pandoc}{}

\section{Leisure-time activities}

\cvitem{Sports}{{\small{}Running, cycling, HIIT, hiking}}
\cvitem{Cooking}{}
\cvitem{Gardening}{}


\section{Professional interests}

\cvitem{Renewable energiy}{Photovoltaics, batteries, thermal systems, eMobility, sector coupling}

\cvitem{Software development}{
  FOSS/Linux development, 
  Trade literature, 
  Hackathons, 
  SoCraTes, 
  Functional software architecture
}

\section{Publications}

\cvitem{Conference paper}{Jakobi, M.; Kunath, L.; Witzig, A. \emph{\href{https://proceedings.ises.org/paper/eurosun2018/eurosun2018-0147-Jakobi.pdf}{BIM use-case: Model-based performance optimization.}} EuroSun international conference on solar energy for buildings and industry, Rapperswil, 2018.}

\cvitem{Conference paper}{Jakobi, M.; Stöckli, U.; Tjaden, T.; Quaschning, Q. \emph{\href{http://proceedings.ises.org/paper/eurosun2018/eurosun2018-0151-Jakobi.pdf}{From simulation to reality: IEC 61499 compliant control applications for solar energy systems.}} EuroSun international conference on solar energy for buildings and industry, Rapperswil, 2018.}

\cvitem{Conference paper}{Jakobi, M.; Stöckli, U.; Tjaden, T.; Quaschning, Q. \emph{\href{https://www.velasolaris.com/wp-content/uploads/2018/12/20180504_publikation_symposiumpv18_jakobi.pdf}{Von der Simulation zur Realität: IEC 61499 konforme Regelanwendungen für Solare Energiesysteme.}} Symposium photovoltaische Solarenergie, Bad Staffelstein, 2018.}

\cvitem{Thesis}{Jakobi, M.: \emph{\href{https://pvspeicher.htw-berlin.de/wp-content/uploads/2017/11/Jakobi-2017-Development-of-model-based-control-applications-compliant-with-IEC-61499.pdf}{Development of model-based control applications compliant with IEC~61499 for building energy systems with a focus on photovoltaics.}} Master's thesis, Hochschule für
	Technik und Wirtschaft, Berlin, 2017.}

\cvitem{Thesis}{Jakobi, M.: \emph{\href{http://pvspeicher.htw-berlin.de/wp-content/uploads/2014/05/JAKOBI-2015-Optimierung-des-Netzeinspeiseverhaltens-von-deutschlandweit-verteilten-PV-Speichersytemen-mit-prognosebasierten-Betriebsstrategien.pdf}{\textit{Optimierung des Netzeinspeiseverhaltens von deutschlandweit verteilten PV-Speichersystemen mit prognosebasierten Betriebsstrategien.}}} Bachelor's thesis, Hochschule für
Technik und Wirtschaft, Berlin, 2016.}

\cvitem{Data \& code}{Jakobi, M.; Schmidt, M.; Anyangbe, F. \emph{\href{https://github.com/MrcJkb/lfpBattery/blob/master/Documentation/lfpBattery_Documentation.pdf}{Cell resolved Matlab OOP model of a lithium iron phosphate battery pack}}, TU Berlin, 2017.}

\cvitem{Co-author}{Weniger, J.; Bergner, J.; Beier, D.; Jakobi, M.; Tjaden, T.; Quaschning,
Q.: \emph{\href{https://pvspeicher.htw-berlin.de/wp-content/uploads/2014/04/WENIGER-2015-Grid-Feed-in-Behavior-of-Distributed-PV-Battery-Systems.pdf}{Grid Feed-in Behavior of Distributed PV Battery Systems}}.
30th European Photovoltaic Solar Energy Conference and Exhibition,
Hamburg, 2015. }

\end{document}
